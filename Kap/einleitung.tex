\chapter{Einleitung}
Einleitung 
Dennis Gabor entwickelte 1948 die Grundlagen der heutigen Holografie. 
Er schlug vor, kohärente Wellenfelder unter Ausnutzung von Interferenzerscheinungen amplituden- und phasengetreu zu registrieren und später durch Beugung zurückzugewinnen. 
Zur optimalen Umsetzung dieser Idee fehlte jedoch die ideale Lichtquelle, der Laser. 
Anfang der 60er Jahre, kurz nach dessen Entdeckung, erlebte die bis dahin etwas in Vergessenheit geratene Technik der Holografie einen enormen Aufschwung. 
Heute zählt die Holografie wohl zu einem der interessantesten Zweige der modernen Optik. 
Mit ihrer Hilfe können verschiedenartige Messverfahren in eleganter Weise durchgeführt werden, die ohne Holografie unmöglich wären.
